% ----------------------------------------------------------
% Apêndices
% ----------------------------------------------------------

% Inicia os apêndices
\begin{apendicesenv}

% Imprime uma página indicando o início dos apêndices
\partapendices

\chapter{Código do detector facial}\label{cap:anexo_detector_facial_opencv}

\begin{code}
\caption{Detector facial usando a biblioteca OpenCV}
\label{cod:detector_facial_opencv}
\inputminted{python}{codigos/detector_facial-imagem.py}
\end{code}
% ----------------------------------------------------------

\chapter{Código para Raspberry Pi do detector facial usando OpenCV}\label{cap:anexo_detector_raspberry}

\begin{code}
\caption{Detector facial usando OpenCV e picamera}
\label{cod:detector_opencv_picamera}
\inputminted{python}{codigos/detector_facial-picamera.py}
\end{code}
% ----------------------------------------------------------

\chapter{Código usado para gerar gráficos para ilustrar funcionamento do PCA}\label{cap:ilustra_pca}

\begin{code}
\caption{Código gerador dos gráficos da \autoref{fig:pca}}
\label{cod:ilustra_pca}
\inputminted{python}{codigos/ilustra_pca.py}
\end{code}
% ----------------------------------------------------------

\chapter{Código usado para obter imagens da face média e dos componentes principais}\label{cap:pca_opencv}

\begin{code}
\caption{Ilustração }
\label{cod:pca_opencv}
\inputminted{python}{codigos/ilustra_eigenfaces.py}
\end{code}
% ----------------------------------------------------------

\chapter{Código do reconhecedor de faces por Eigenfaces usando OpenCV}\label{cap:eigenfaces_opencv}

\begin{code}
\caption{Reconhecimento facial por Eigenfaces usando OpenCV}
\label{cod:eigenfaces_opencv}
\inputminted{python}{codigos/eigenfaces.py}
\end{code}
% ----------------------------------------------------------

\end{apendicesenv}
