\chapter*{Introdução}
\addcontentsline{toc}{chapter}{Introdução}\markboth{Introdução}{Introdução}

A capacidade de identificar faces e suas emoções é um importante mecanismo neurológico para interações sociais que, em certo grau, está presente até mesmo em recém nascidos \cite{morton1991conspec, fantz1961origin}.

Enquanto humanos possuem uma notável capacidade de reconhecer faces, o desenvolvimento de sistemas computacionais com capacidade similar é uma área de pesquisa em andamento há mais de cinco décadas \cite{bledsoe1964facial, chan1965man, bledsoe1966man, bledsoe1966model, boyer1991biographical, kelly1970visual, kanade1973picture}.

O desenvolvimento de modelos computacionais para reconhecimento facial são de interesse para diversas aplicações práticas como identificação criminal, sistemas de segurança, biometria, processamento de imagens e interação humano-computador.

Infelizmente, o desenvolvimento de um sistema computacional para reconhecimento facial automatizado é bastante difícil por diversos motivos. Faces são complexas, multidimensionais e expressivas \cite{turk1991eigenfaces} e as imagens podem sofrer variações em escala, localização, ponto de visão, iluminação e obstrução \cite{censtudy}.

\section*{Motivação}\label{sec:motivacao}
\addcontentsline{toc}{section}{\protect\numberline{}Motivação}

O desenvolvimento de novos algoritmos de processamento de imagens acompanhado do maior acesso a câmeras digitais, hardwares com alto poder de processamento, bibliotecas para visão computacional, como a OpenCV, e APIs de análise de imagem, como o Google Vision, o Microsoft Cognitive Services, o Amazon Rekognition e o IBM Watson Visual Recognition, tornou viável a implantação de sistemas com detecção e reconhecimento facial em diversas empresas e produtos.

Nos últimos anos, lojas de varejo têm usado com sucesso as tecnologias de detecção e reconhecimento facial para segurança, personalização de serviço, marketing e análise de sentimento \cite{fortune2015walmart, consumer2015facial, exame2018pontofrio, 2017recfacial, 2015bbcfacewatch}.

Este trabalho explora um algoritmo de grande impacto na última década, o algoritmo Viola-Jones, e foi motivado pela implementação de um sistema de reconhecimento facial em uma rede de lojas de varejo na cidade do Rio de Janeiro.

\section*{Objetivos}\label{sec:objetivos}
\addcontentsline{toc}{section}{\protect\numberline{}Objetivos}

O objetivo deste trabalho é fornecer uma base teórica e prática para a futura construção de um sistema de detecção e reconhecimento facial, que será utilizado em uma loja de varejo e, possivelmente, em outros estabelecimentos.

O sistema de detecção será executado em um computador de placa única com câmera, como o Raspberry Pi, que ficará instalado na saída da loja.
Ele capturará fotos continuamente e deverá ser capaz de identificar quais fotos contém faces, extrair essas faces e enviá-las para o servidor no qual os algoritmos de reconhecimento facial serão executados.

O presente trabalho deverá identificar um algoritmo de detecção facial capaz de ser executado quase em tempo real nesse hardware com recursos limitados e também encontrar ou desenvolver uma implementação do algoritmo de detecção facial compatível com a arquitetura ARM.