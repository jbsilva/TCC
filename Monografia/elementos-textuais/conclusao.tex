% ----------------------------------------------------------
% Conclusão (outro exemplo de capítulo sem numeração e presente no sumário)
% ----------------------------------------------------------
% \chapter*[Conclusão]{Conclusão}
% \addcontentsline{toc}{chapter}{Conclusão}
% ----------------------------------------------------------

O presente trabalho apresentou uma visão ampla dos conceitos e métodos de detecção e reconhecimento facial e explicações detalhadas sobre dois algoritmos, incluindo a implementação de sistemas computacionais capazes de detectar e reconhecer faces utilizando esses algoritmos.

Para alcançar o objetivo de fornecer uma base teórica para a implantação de um sistema de detecção e reconhecimento facial, foi necessária uma extensa pesquisa bibliográfica e a conclusão de cursos sobre visão computacional.

Ficou claro que reconhecimento facial irrestrito ainda é um desafio na área de processamento de imagens e precisa superar diversas dificuldades geradas por fatores como iluminação, baixa resolução, variação de pose, oclusão e expressão facial.

Foi possível mostrar a validade do algoritmo de Viola e Jones utilizando a biblioteca OpenCV.

Dificuldades do treinamento: determinar taxas de detecção e falso-positivo, determinar número de estágios, tempo de treinamento x performance do detector.

\section{Sugestão de melhorias e trabalhos futuros}

Treinar um novo detector utilizando mais imagens positivas e negativas melhor selecionadas.

Usar valor mais baixo para taxa máxima de falso-positivo (maxFalseAlarmRate)

Variar taxas de detecção e falso-positivo por estágio da cascata.

Usar mais estágios

Paralelização do processo

Uso de arquitetura na nuvem como Amazon EC2

Imprimir case em 3D para Raspberry Pi

Utilizar algoritmos baseados em redes neurais ou soluções comerciais