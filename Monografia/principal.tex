% ----------------------------------------------------------
% VERSÃO ORIGINAL
% ----------------------------------------------------------
% The Current Maintainer of this work is the abnTeX2 team, led
% by Lauro César Araujo. Further information are available on
% http://abntex2.googlecode.com/

\documentclass[
	% -- opções da classe memoir --
	12pt,				% tamanho da fonte
	openright,			% capítulos começam em pág ímpar (insere página vazia caso preciso)
	oneside,			% para impressão em verso e anverso coloque twoside
	a4paper,			% tamanho do papel.
	% -- opções da classe abntex2 --
	%chapter=TITLE,		% títulos de capítulos convertidos em letras maiúsculas
	%section=TITLE,		% títulos de seções convertidos em letras maiúsculas
	%subsection=TITLE,	% títulos de subseções convertidos em letras maiúsculas
	%subsubsection=TITLE,% títulos de subsubseções convertidos em letras maiúsculas
	% -- opções do pacote babel --
	french,				% idioma adicional para hifenização
	spanish,			% idioma adicional para hifenização
	english,			% idioma adicional para hifenização
	brazil				% o último idioma é o principal do documento
	]{abntex2}


% ----------------------------------------------------------
% PACOTES
% ----------------------------------------------------------
% ----------------------------------------------------------
% PACOTES BÁSICOS
% ----------------------------------------------------------
\usepackage{lmodern}            % Usa a fonte Latin Modern          
\usepackage[T1]{fontenc}        % Selecao de codigos de fonte.
\usepackage[utf8]{inputenc}     % Codificacao do documento (conversão automática dos acentos)
\usepackage{lastpage}           % Usado pela Ficha catalográfica
\usepackage{indentfirst}        % Indenta o primeiro parágrafo de cada seção.
\usepackage{color}              % Controle das cores
\usepackage{graphicx}           % Inclusão de gráficos
\usepackage{microtype}          % para melhorias de justificação
\usepackage[newfloat]{minted}   % Pygments
\usepackage{nomencl}            % Necessário para o commando makeindex
\usepackage[brazilian,hyperpageref]{backref}    % Paginas com as citações na bibl
\usepackage[alf]{abntex2cite}   % Citações padrão ABNT
\usepackage{lipsum}             
\usepackage{adjustbox}
\usepackage{amsmath,amssymb}
\usepackage{siunitx}
\usepackage{booktabs}
\usepackage{longtable}
\usepackage{tabu}
\usepackage{multirow}
\usepackage{subcaption}
%\usepackage{epigraph}
\usepackage{lscape}
\usepackage{xurl}               % Quebra links longos direito
\usepackage[ruled,vlined,portuguese,onelanguage]{algorithm2e} %for psuedo code
\usepackage{tikz}
\usetikzlibrary{arrows,automata,calc,chains,fit,matrix,positioning,quotes,shadows,shapes}
\usepackage{pgfplots}
\pgfplotsset{compat=newest,compat/show suggested version=false}

% CONFIGURAÇÕES DE PACOTES
% Configurações do pacote backref
% Usado sem a opção hyperpageref de backref
\renewcommand{\backrefpagesname}{Citado na(s) página(s):~}
% Texto padrão antes do número das páginas
\renewcommand{\backref}{}
% Define os textos da citação
\renewcommand*{\backrefalt}[4]{
    \ifcase #1 %
        Nenhuma citação no texto.%
    \or
        Citado na página #2.%
    \else
        Citado #1 vezes nas páginas #2.%
    \fi}%

% Corrige bug do anexo (https://github.com/abntex/abntex2/issues/76)
\newcommand{\refanexo}[1]{\hyperref[#1]{Anexo~\ref{#1}}}

%%%%%%%%%%%%%%%%%%%%%%%%%%%%%%%%%%%%%%%%%%%%%%%%%%%%%%%%%%%%%%%%%%%%%%%%
%% TIKZ
%%%%%%%%%%%%%%%%%%%%%%%%%%%%%%%%%%%%%%%%%%%%%%%%%%%%%%%%%%%%%%%%%%%%%%%%
% Tabelas
\tikzset{square matrix/.style={
    matrix of nodes,
    column sep=-\pgflinewidth, row sep=-\pgflinewidth,
    nodes={draw,
      minimum height=#1,
      anchor=center,
      text width=#1,
      align=center,
      inner sep=0pt
    },
  },
  square matrix/.default=2em
}

% Distância entre extremidades de dois nós
\makeatletter
\def\DistanciaExtremidades(#1,#2)#3{%
\pgfpointdiff{\pgfpointanchor{#1}{west}}{\pgfpointanchor{#2}{east}}
\pgfmathsetmacro{\myheight}{veclen(\pgf@x,\pgf@y)}
\global\expandafter\edef\csname #3\endcsname{\myheight}
}
\makeatother

% Distância entre centros de dois nós
\makeatletter
\def\DistanciaCentros(#1,#2)#3{%
\pgfpointdiff{\pgfpointanchor{#1}{center}}{\pgfpointanchor{#2}{center}}
\pgfmathsetmacro{\myheight}{veclen(\pgf@x,\pgf@y)}
\global\expandafter\edef\csname #3\endcsname{\myheight}
}
\makeatother

%%%%%%%%%%%%%%%%%%%%%%%%%%%%%%%%%%%%%%%%%%%%%%%%%%%%%%%%%%%%%%%%%%%%%%%%
%% LISTA DE CÓDIGOS
%%%%%%%%%%%%%%%%%%%%%%%%%%%%%%%%%%%%%%%%%%%%%%%%%%%%%%%%%%%%%%%%%%%%%%%%
\newenvironment{code}{\captionsetup{type=listing}}{}

\makeatletter
\let\l@listing\l@figure
\def\newfloat@listoflisting@hook{\let\figurename\listingname}
\makeatother

\SetupFloatingEnvironment{listing}{%
  fileext=lol,
  listname={Lista de códigos},
  name=Código,
  placement=p,
  within=none,
  chapterlistsgaps=on}

\setminted{% bgcolor = gray!15, 
            frame = lines,
            mathescape,
            autogobble,
            %breakanywhere,
            breaklines,
            framesep = 2mm,
            baselinestretch = 1.2,
            fontsize = \footnotesize,
            linenos
}

%%%%%%%%%%%%%%%%%%%%%%%%%%%%%%%%%%%%%%%%%%%%%%%%%%%%%%%%%%%%%%%%%%%%%%%%
%% LISTA DE ALGORITMOS
%%%%%%%%%%%%%%%%%%%%%%%%%%%%%%%%%%%%%%%%%%%%%%%%%%%%%%%%%%%%%%%%%%%%%%%%
%\let\oldlistofalgorithms\listofalgorithms
%\let\oldnumberline\numberline%
%\newcommand{\algnumberline}[1]{Algoritmo~#1 -- }
%\renewcommand{\listofalgorithms}{%
%  \let\numberline\algnumberline%
%  \oldlistofalgorithms
%  \let\numberline\oldnumberline%
%}


% ----------------------------------------------------------
% CAPA E FOLHA DE ROSTO
% ----------------------------------------------------------
% ----------------------------------------------------------
% CAPA E FOLHA DE ROSTO
% ----------------------------------------------------------
\titulo{Explorando o algoritmo de Viola-Jones na detecção e reconhecimento facial}
\autor{Julio Batista Silva}
\local{São Carlos, Brasil}
\data{2018}
\orientador{Prof.~Dr.~ Alexandre Luis Magalhães Levada}
\coorientador{}
\instituicao{%
  Universidade Federal de São Carlos -- UFSCar
  \par
  Departamento de Computação
  \par
  Engenharia de Computação}
\tipotrabalho{Trabalho de Conclusão de Curso}
% O preambulo deve conter o tipo do trabalho, o objetivo, 
% o nome da instituição e a área de concentração 
\preambulo{Trabalho de conclusão de curso.}



% ----------------------------------------------------------
% CONFIGURAÇÕES
% ----------------------------------------------------------

% Configurações de aparência do PDF final

% alterando o aspecto da cor azul
\definecolor{blue}{RGB}{41,5,195}

% informações do PDF
\makeatletter
\hypersetup{
        %pagebackref=true,
        pdftitle={\@title},
        pdfauthor={\@author},
        pdfsubject={\imprimirpreambulo},
        pdfcreator={LaTeX with abnTeX2},
        pdfkeywords={abnt}{latex}{abntex}{abntex2}{trabalho acadêmico},
        colorlinks=true,            % false: boxed links; true: colored links
        linkcolor=blue,             % color of internal links
        citecolor=blue,             % color of links to bibliography
        filecolor=magenta,              % color of file links
        urlcolor=blue,
        bookmarksdepth=4
}
\makeatother

% Espaçamentos entre linhas e parágrafos
% O tamanho do parágrafo é dado por:
\setlength{\parindent}{1.3cm}

% Controle do espaçamento entre um parágrafo e outro:
\setlength{\parskip}{0.2cm}  % tente também \onelineskip

% compila o indice
\makeindex
\makenomenclature

% ----------------------------------------------------------
% INÍCIO DOCUMENTO
% ----------------------------------------------------------
\begin{document}

% Retira espaço extra obsoleto entre as frases.
\frenchspacing

% ----------------------------------------------------------
% ELEMENTOS PRÉ-TEXTUAIS
% ----------------------------------------------------------
% \pretextual

% Capa
\imprimircapa

% Folha de rosto
% (o * indica que haverá a ficha bibliográfica)
\imprimirfolhaderosto*

% ----------------------------------------------------------
% FICHA BIBLIOGRAFICA
% ----------------------------------------------------------

% Isto é um exemplo de Ficha Catalográfica, ou ``Dados internacionais de
% catalogação-na-publicação''. Você pode utilizar este modelo como referência. 
% Porém, provavelmente a biblioteca da sua universidade lhe fornecerá um PDF
% com a ficha catalográfica definitiva após a defesa do trabalho. Quando estiver
% com o documento, salve-o como PDF no diretório do seu projeto e substitua todo
% o conteúdo de implementação deste arquivo pelo comando abaixo:
%
% \begin{fichacatalografica}
%     \includepdf{fig_ficha_catalografica.pdf}
% \end{fichacatalografica}
\begin{fichacatalografica}
	\vspace*{\fill}					% Posição vertical
	\hrule							% Linha horizontal
	\begin{center}					% Minipage Centralizado
	\begin{minipage}[c]{12.5cm}		% Largura
	
	\imprimirautor
	
	\hspace{0.5cm} \imprimirtitulo  / \imprimirautor. --
	\imprimirlocal, \imprimirdata-
	
	\hspace{0.5cm} \pageref{LastPage} p. : il. (algumas color.) ; 30 cm.\\
	
	\hspace{0.5cm} \imprimirorientadorRotulo~\imprimirorientador\\
	
	\hspace{0.5cm}
	\parbox[t]{\textwidth}{\imprimirtipotrabalho~--~\imprimirinstituicao,
	\imprimirdata.}\\
	
	\hspace{0.5cm}
		1. Detecção facial.
		2. Viola-Jones.
		I. Alexandre Luiz Magalhães Levada.
		II. Universidade Federal de São Carlos.
		III. Departamento de Computação.
		IV. Explorando o algoritmo de Viola-Jones na detecção e reconhecimento facial\\ 			
	
	\hspace{8.75cm} CDU 02:141:005.7\\
	
	\end{minipage}
	\end{center}
	\hrule
\end{fichacatalografica}

%% ----------------------------------------------------------
% INSERIR ERRATA
% ----------------------------------------------------------
\begin{errata}
Elemento opcional da \citeonline[4.2.1.2]{NBR14724:2011}. Exemplo:

\vspace{\onelineskip}

FERRIGNO, C. R. A. \textbf{Tratamento de neoplasias ósseas apendiculares com
reimplantação de enxerto ósseo autólogo autoclavado associado ao plasma
rico em plaquetas}: estudo crítico na cirurgia de preservação de membro em
cães. 2011. 128 f. Tese (Livre-Docência) - Faculdade de Medicina Veterinária e
Zootecnia, Universidade de São Paulo, São Paulo, 2011.

\begin{table}[htb]
\center
\footnotesize
\begin{tabular}{|p{1.4cm}|p{1cm}|p{3cm}|p{3cm}|}
  \hline
   \textbf{Folha} & \textbf{Linha}  & \textbf{Onde se lê}  & \textbf{Leia-se}  \\
    \hline
    1 & 10 & auto-conclavo & autoconclavo\\
   \hline
\end{tabular}
\end{table}

\end{errata}


% ----------------------------------------------------------
% INSERIR FOLHA DE APROVAÇÃO
% ----------------------------------------------------------
% Isto é um exemplo de Folha de aprovação, elemento obrigatório da NBR
% 14724/2011 (seção 4.2.1.3). Você pode utilizar este modelo até a aprovação
% do trabalho. Após isso, substitua todo o conteúdo deste arquivo por uma
% imagem da página assinada pela banca com o comando abaixo:
%
% \includepdf{folhadeaprovacao_final.pdf}
%
\begin{folhadeaprovacao}

  \begin{center}
    {\ABNTEXchapterfont\large\imprimirautor}

    \vspace*{\fill}\vspace*{\fill}
    \begin{center}
      \ABNTEXchapterfont\bfseries\Large\imprimirtitulo
    \end{center}
    \vspace*{\fill}
    
    \hspace{.45\textwidth}
    \begin{minipage}{.5\textwidth}
        \imprimirpreambulo
    \end{minipage}%
    \vspace*{\fill}
   \end{center}
        
   Trabalho aprovado. \imprimirlocal, 12 de julho de 2018:

   \assinatura{\textbf{\imprimirorientador} \\ Orientador} 
   \assinatura{\textbf{Prof.~Dr.~Fredy João Valente} \\ Membro da banca}
   \assinatura{\textbf{Prof\textsuperscript{a}.~Dr\textsuperscript{a}.~Marcela Xavier Ribeiro} \\ Membro da banca}
   %\assinatura{\textbf{Professor} \\ Convidado 3}
      
   \begin{center}
    \vspace*{0.5cm}
    {\large\imprimirlocal}
    \par
    {\large\imprimirdata}
    \vspace*{1cm}
  \end{center}
  
\end{folhadeaprovacao}


\begin{dedicatoria}

    Dedico este trabalho aos meus pais, Cesar de Souza e Silva e Fátima Aparecida Batista Silva, por todo o amor, incentivo aos estudos e investimento, que foram fundamentais para a realização deste trabalho.

\end{dedicatoria}

% ----------------------------------------------------------
% AGRADECIMENTOS
% ----------------------------------------------------------
\begin{agradecimentos}
    Agradeço aos meus pais, Cesar e Fátima, que sempre me apoiaram e me ajudaram durante a graduação.
    
    À minha companheira, Louise Lobão, pelo amor e carinho, que me deram força para superar as dificuldades da vida.
    
    Ao meu orientador, Alexandre Levada, pelo apoio, atenção, paciência e conselhos.
    
    À Universidade Federal de São Carlos por oferecer recursos e acesso a grandes professores, que foram fundamentais para a minha formação.
    
    Aos meus veteranos, calouros e colegas de curso que, de alguma forma, contribuíram para a minha graduação.
    
    Aos amigos que fiz durante o intercâmbio na Alemanha.
    
    À CAPES pela bolsa que me sustentou durante o intercâmbio.
    
    Ao meu gestor, Felipe Silva, por ter sugerido o tema deste trabalho e me ajudado a equilibrar meu tempo entre trabalho e estudo.
    
    A todos da Visagio.
    
    Ao pessoal do Dragões de Garagem.

    Aos criadores do abn\TeX\ por deixar este trabalho tão bonito e bem formatado.

    Aos meus gatos, Amélia e Joaquim, que me fizeram companhia durante a escrita desta monografia.

\end{agradecimentos}


% ----------------------------------------------------------
% EPÍGRAFE
% ----------------------------------------------------------
\begin{epigrafe}
    \vspace*{\fill}
	\begin{flushright}
		\textit{``Miau, \\
		Miau, \\
		Miau. \\
		(Joaquim e Amélia, 2018)}
	\end{flushright}
\end{epigrafe}


% ----------------------------------------------------------
% RESUMOS
% ----------------------------------------------------------

% resumo em português
\setlength{\absparsep}{18pt} % ajusta o espaçamento dos parágrafos do resumo
\begin{resumo}
As inúmeras aplicações das técnicas para detecção e reconhecimento facial têm chamado muita atenção de empresas e governos. Esse crescente interesse pelo assunto atraiu investimentos em pesquisas e resultou em progressos significantes no desenvolvimento de novos métodos, bibliotecas, produtos e serviços.

Apesar de muitas dessas ferramentas serem descritas como simples e utilizáveis sem a necessidade de conhecimentos prévios em visão computacional, um embasamento teórico permite escolher as tecnologias apropriadas e usá-las de forma otimizada, considerando suas capacidades e limitações.

Este trabalho introduz as áreas de detecção e reconhecimento facial através de uma extensa revisão bibliográfica, que fornece uma visão geral sobre inúmeros métodos encontrados na literatura e apresenta uma coletânea de recursos úteis ao treinamento de classificadores e validação dos algoritmos.

Também é feito um estudo mais aprofundado sobre Viola-Jones e Eigenfaces ao apresentar o projeto e a implementação de um sistema capaz de detectar e reconhecer faces construído através da combinação desses métodos. É mostrado que a taxa de detecção do Eigenfaces não é suficiente para o uso desejado e são propostas alternativas.

O primeiro módulo desse sistema é executado em um Raspberry Pi e é um exemplo de como aliar conhecimento teórico, bibliotecas open source, ferramentas comerciais e hardware para a criação de um produto lucrativo.

 \textbf{Palavras-chaves}: Detecção facial. Viola-Jones. Reconhecimento facial. Eigenfaces. Raspberry Pi.
\end{resumo}

% resumo em inglês
\begin{resumo}[Abstract]
 \begin{otherlanguage*}{english}

The numerous applications of facial detection and recognition techniques have attracted much attention from companies and governments. This growing interest in the subject attracted investment in research and resulted in significant advances in the development of new methods, libraries, products and services.

Although many of these tools are described as simple and usable without the need for prior knowledge in computer vision, a theoretical basis allows one to choose the appropriate technologies and use them optimally, considering their capabilities and limitations.

This work introduces the areas of facial detection and recognition through an extensive bibliographic review, which provides an overview of the numerous methods found in the literature and presents a collection of useful resources for the training of classifiers and validation of algorithms.

Further study is also made on Viola-Jones and Eigenfaces while presenting the design and implementation of a system capable of detecting and recognizing faces constructed by combining these methods. It is shown that the detection rate of the Eigenfaces is not sufficient for the desired use and alternative solutions are proposed.

The first module of this system runs on a Raspberry Pi and is an example of how to combine theoretical knowledge, open source libraries, comercial tools and hardware in the creation of a profitable product.
   \vspace{\onelineskip}
 
   \noindent 
   \textbf{Key-words}: Facial detection. Viola-Jones. Facial recognition. Eigenfaces. Raspberry Pi.
 \end{otherlanguage*}
\end{resumo}

% resumo em francês 
%\begin{resumo}[Résumé]
% \begin{otherlanguage*}{french}
%    Il s'agit d'un résumé en français.
 
%   \textbf{Mots-clés}: latex. abntex. publication de textes.
% \end{otherlanguage*}
%\end{resumo}

% resumo em espanhol
%\begin{resumo}[Resumen]
% \begin{otherlanguage*}{spanish}
%   Este es el resumen en español.
  
%   \textbf{Palabras clave}: latex. abntex. publicación de textos.
% \end{otherlanguage*}
%\end{resumo}


% ----------------------------------------------------------
% EPÍGRAFE
% ----------------------------------------------------------
\begin{epigrafe}
    \vspace*{\fill}
	\begin{flushright}
		\textit{``Miau, \\
		Miau, \\
		Miau. \\
		(Joaquim e Amélia, 2018)}
	\end{flushright}
\end{epigrafe}




% ----------------------------------------------------------
% inserir lista de ilustrações
% ----------------------------------------------------------
\pdfbookmark[0]{\listfigurename}{lof}
\listoffigures*
\cleardoublepage

% ----------------------------------------------------------
% inserir lista de tabelas
% ----------------------------------------------------------
\pdfbookmark[0]{\listtablename}{lot}
\listoftables*
\cleardoublepage

% ----------------------------------------------------------
% inserir lista siglas e abreviaturas
% ----------------------------------------------------------
% ----------------------------------------------------------
% INSERIR LISTA DE ABREVIATURAS E SIGLAS
% ----------------------------------------------------------
\begin{siglas}
  \item[FERET] Face Recognition Technology
  \item[RAM] Random Access Memory
\end{siglas}



% ----------------------------------------------------------
% inserir lista símbolos
% ----------------------------------------------------------
% INSERIR LISTA DE SÍMBOLOS
% ----------------------------------------------------------
\begin{simbolos}
  \item[$ \Gamma $] Letra grega Gama
  \item[$ \Lambda $] Lambda
  \item[$ \zeta $] Letra grega minúscula zeta
  \item[$ \in $] Pertence
\end{simbolos}


% ----------------------------------------------------------

% ----------------------------------------------------------
% inserir o sumario
% ----------------------------------------------------------
\pdfbookmark[0]{\contentsname}{toc}
\tableofcontents*
\cleardoublepage

% ----------------------------------------------------------
% ELEMENTOS TEXTUAIS
% ----------------------------------------------------------
\textual

% ----------------------------------------------------------
% Introdução (exemplo de capítulo sem numeração, mas presente no Sumário)
% ----------------------------------------------------------
\chapter*[Introdução]{Introdução}
\addcontentsline{toc}{chapter}{Introdução}
% ----------------------------------------------------------
% ----------------------------------------------------------
% Introdução (exemplo de capítulo sem numeração, mas presente no Sumário)
% ----------------------------------------------------------
% \chapter*[Introdução]{Introdução}
% \addcontentsline{toc}{chapter}{Introdução}
% ----------------------------------------------------------

%\chapter{Introdução}

A capacidade de identificar faces e suas emoções é um importante mecanismo neurológico para interações sociais que, em certo grau, está presente até mesmo em recém nascidos \cite{morton1991conspec} \cite{fantz1961origin}.

Enquanto humanos possuem uma notável capacidade de reconhecer faces, o desenvolvimento de sistemas computacionais com capacidade similar é uma área de pesquisa em andamento há mais de cinco décadas \cite{bledsoe1964facial} \cite{chan1965man} \cite{bledsoe1966man} \cite{bledsoe1966model} \cite{boyer1991biographical} \cite{kelly1970visual} \cite{kanade1973picture}.

O desenvolvimento de modelos computacionais para reconhecimento facial são de interesse para diversas aplicações práticas como identificação criminal, sistemas de segurança, biometria, processamento de imagens e interação humano-computador.

Infelizmente, o desenvolvimento de um sistema computacional para reconhecimento facial automatizado é bastante difícil por diversos motivos. Faces são complexas, multidimensionais e expressivas \cite{turk1991eigenfaces} e as imagens podem sofrer variações em escala, localização, ponto de visão, iluminação e obstrução \cite{censtudy}.

\section{Motivação}\label{sec:motivacao}

O desenvolvimento de novos algoritmos de processamento de imagens acompanhado do maior acesso a câmeras digitais, hardwares com alto poder de processamento, bibliotecas para visão computacional, como a OpenCV, e APIs de análise de imagem, como o Google Vision, o Microsoft Cognitive Services, o Amazon Rekognition e o IBM Watson Visual Recognition, tornou viável a implantação de sistemas com detecção e reconhecimento facial em diversas empresas e produtos.

Nos últimos anos, lojas de varejo têm usado com sucesso as tecnologias de detecção e reconhecimento facial para segurança, personalização de serviço, marketing e análise de sentimento \cite{fortune2015walmart} \cite{exame2018pontofrio} \cite{2017recfacial}.

Este trabalho explora um algoritmo de grande impacto na última década, o algoritmo Viola-Jones, e foi motivado pela implementação de um sistema de reconhecimento facial em uma rede de lojas de varejo na cidade do Rio de Janeiro.

\section{Objetivos}\label{sec:objetivos}

O objetivo deste trabalho é fornecer uma base teórica para uma futura implantação de um sistema de reconhecimento facial, através do estudo do algoritmo de Viola-Jones.


% ----------------------------------------------------------
% PARTE
% ----------------------------------------------------------
\part{}

% ----------------------------------------------------------
% Capitulo com exemplos de comandos inseridos de arquivo externo
% ----------------------------------------------------------
\chapter{Detecção facial}\label{cap:detecao_facial}

\chapterprecis{Isto é uma sinopse de capítulo.}\index{sinopse de capítulo}

O papel de um detector de faces é, dada uma imagem arbitrária, determinar se ela contém faces humanas ou não e, caso positivo, retornar a localização e dimensões de cada face \cite{censtudy}.

A detecção de faces é utilizada em câmeras fotográficas para ajuste automático de foco, em softwares de imagens e redes sociais para marcação de pessoas e é uma etapa importante para o processo de reconhecimento facial. Antes de executar um algoritmo de reconhecimento facial, é de praxe realizar uma detecção facial a fim de concentrar os esforços do reconhecedor facial apenas nas áreas relevantes.

É preciso minimizar tanto a quantidade de faces não identificadas (falso-negativos) quanto objetos reconhecidos erroneamente como faces (falso-positivos) para obter uma performance aceitável. Para tanto, algoritmos de aprendizado de máquina podem ser muito úteis.

Diversas dificuldades influenciam na eficiência dos algoritmos, como ruídos, variação de iluminação, expressões faciais, imagem de fundo, orientação da cabeça, obstrução da face ou sobreposição de faces \cite{de2015processo} e, nota-se que, para análise de streamings de vídeo, é fundamental que a detecção facial seja realizada em tempo real.

Segundo, \citeonline{de2015processo}, "as técnicas mais citadas para realizar a detecção de faces são: casamento de padrões que consiste na detecção por meio de comparações com formas ométricas, modelos estatísticos, modelos baseado em redes neurais, modelos baseados em tons de pele e o Viola; Jones". Dentre os trabalhos anteriores a Viola-Jones, destacam-se \citeonline{rowley1998neural} e \citeonline{schneiderman2000statistical}.


% ----------------------------------------------------------
% PARTE
% ----------------------------------------------------------
\part{Dicas de úteis}

% ----------------------------------------------------------
% Capitulo com exemplos de comandos inseridos de arquivo externo
% ----------------------------------------------------------
\chapter{Dicas úteis}\label{cap_dicas}

O presente capítulo foi escrito foi Roney (https://github.com/roneyfraga), que não faz parte da equipe da Equipe \abnTeX . Apenas algumas dicas serão acrescentadas.

\section{Aprender \LaTeX }
Se você não sabe por onde começar a estudar para aprender \LaTeX, segue lista de materiais que eu utilizei e que sempre recorro quando preciso:

\begin{itemize}
    \item \href{http://www.mat.ufmg.br/~regi/topicos/intlat.pdf}{Introdução ao \LaTeX} do Professor Reginaldo J. Santos.
    \item \href{http://zelmanov.ptep-online.com/ctan/lshort_port.pdf}{Uma introdução não tão pequena de \LaTeX} por Tobias Oetiker Hubert Partl, Irene Hyna e Elisabeth Schlegl. Tradução de Alberto Simões.
    \item \href{http://www.uel.br/projetos/matessencial/superior/pdfs/latexmat.pdf}{\LaTeX\ para matemática com o TeXnicCenter} de Ulysses Sodré.
    \item \href{http://en.wikibooks.org/wiki/LaTeX}{\LaTeX\ Wikibooks.}
\end{itemize}

\section{\LaTeX + R}

\begin{itemize}
    \item knitr
    \item tikz
    \item tables
    \item xtable
\end{itemize}

\section{\LaTeX + Referências}

\begin{itemize}
    \item zotero
    \item mendley
    \item google
    \item sites de revistas
\end{itemize}

\section{\LaTeX + diversos}


% ----------------------------------------------------------

% Finaliza a parte no bookmark do PDF
% para que se inicie o bookmark na raiz
% e adiciona espaço de parte no Sumário
% ----------------------------------------------------------
\phantompart

% ----------------------------------------------------------
% Conclusão (outro exemplo de capítulo sem numeração e presente no sumário)
% ----------------------------------------------------------
\chapter*[Conclusão]{Conclusão}
\addcontentsline{toc}{chapter}{Conclusão}
% ----------------------------------------------------------
\chapter{Conclusão}

Este trabalho mostrou...

As principais contribuições deste trabalho são...


% ----------------------------------------------------------
% ELEMENTOS PÓS-TEXTUAIS
% ----------------------------------------------------------
\postextual
% ----------------------------------------------------------

% ----------------------------------------------------------
% Referências bibliográficas
% ----------------------------------------------------------
\bibliography{elementos-postextuais/referencias}

% ----------------------------------------------------------
% Glossário
% ----------------------------------------------------------
%
% Consulte o manual da classe abntex2 para orientações sobre o glossário.
%
%\glossary

% ----------------------------------------------------------
% Apêndices
% ----------------------------------------------------------
% ----------------------------------------------------------
% Apêndices
% ----------------------------------------------------------

% Inicia os apêndices
\begin{apendicesenv}

% Imprime uma página indicando o início dos apêndices
\partapendices

\chapter{Código do detector facial}\label{cap:anexo_detector_facial_opencv}

\begin{code}
\caption{Detector facial usando a biblioteca OpenCV}
\label{cod:detector_facial_opencv}
\inputminted{python}{codigos/detector_facial.py}
\end{code}
% ----------------------------------------------------------

\chapter{Código para Raspberry Pi do detector facial usando OpenCV}\label{cap:anexo_detector_raspberry}

\begin{code}
\caption{Detector facial usando OpenCV e picamera}
\label{cod:detector_opencv_picamera}
\inputminted{python}{codigos/detector_facial_picamera.py}
\end{code}
% ----------------------------------------------------------

\chapter{Código usado para gerar gráficos para ilustrar funcionamento do PCA}\label{cap:ilustra_pca}

\begin{code}
\caption{Código gerador dos gráficos da \autoref{fig:pca}}
\label{cod:ilustra_pca}
\inputminted{python}{codigos/eigenfaces.py}
\end{code}
% ----------------------------------------------------------

\chapter{Código usado para obter imagens da face média e dos componentes principais}\label{cap:pca_opencv}

\begin{code}
\caption{Ilustração }
\label{cod:pca_opencv}
\inputminted{python}{codigos/eigenfaces.py}
\end{code}
% ----------------------------------------------------------

\chapter{Código do reconhecedor de faces por Eigenfaces usando OpenCV}\label{cap:eigenfaces_opencv}

\begin{code}
\caption{Reconhecimento facial por Eigenfaces usando OpenCV}
\label{cod:eigenfaces_opencv}
\inputminted{python}{codigos/eigenfaces.py}
\end{code}
% ----------------------------------------------------------

\end{apendicesenv}



% ----------------------------------------------------------
% Anexos
% ----------------------------------------------------------
% ----------------------------------------------------------
% Anexos
% ----------------------------------------------------------
% Inicia os anexos
\begin{anexosenv}

% Imprime uma página indicando o início dos anexos
\partanexos

\chapter{Código do AdaBoost}\label{cap:anexo_adaboost_python}

\begin{code}
\caption{AdaBoost. Fonte: \cite{datta2015face}}
\label{cod:adaboost}
\inputminted{python}{codigos/adaboost.py}
\end{code}

\end{anexosenv}


%---------------------------------------------------------------------
% INDICE REMISSIVO
%---------------------------------------------------------------------
\phantompart
\printindex
%---------------------------------------------------------------------

\end{document}
