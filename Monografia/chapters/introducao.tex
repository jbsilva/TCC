\chapter{Introdução}

A capacidade de identificar faces e suas emoções é um importante mecanismo neurológico para interações sociais que, em certo grau, está presente até mesmo em recém nascidos \cite{morton1991conspec} \cite{fantz1961origin}.

Enquanto humanos possuem uma notável capacidade de reconhecer faces, o desenvolvimento de sistemas computacionais com capacidade similar é uma área de pesquisa em andamento há mais de cinco décadas \cite{bledsoe1964facial} \cite{chan1965man} \cite{bledsoe1966man} \cite{bledsoe1966model} \cite{boyer1991biographical} \cite{kelly1970visual} \cite{kanade1973picture}.

O desenvolvimento de modelos computacionais para reconhecimento facial são de interesse para diversas aplicações práticas como identificação criminal, sistemas de segurança, biometria, processamento de imagens e interação humano-computador.

Infelizmente, o desenvolvimento de um sistema computacional para reconhecimento facial automatizado é bastante difícil por diversos motivos. Faces são complexas, multidimensionais e expressivas \cite{turk1991eigenfaces} e as imagens podem sofrer variações em escala, localização, ponto de visão, iluminação e obstrução \cite{censtudy}.

\section{Motivação}

O desenvolvimento de novos algoritmos de processamento de imagens acompanhado do maior acesso a câmeras digitais, hardwares com alto poder de processamento, bibliotecas para visão computacional, como a OpenCV, e APIs de análise de imagem, como o Google Vision, o Microsoft Cognitive Services, o Amazon Rekognition e o IBM Watson Visual Recognition, tornou viável a implantação de sistemas com detecção e reconhecimento facial em diversas empresas e produtos.

Nos últimos anos, lojas de varejo têm usado com sucesso as tecnologias de detecção e reconhecimento facial para segurança, personalização de serviço, marketing e análise de sentimento \cite{fortune2015walmart} \cite{exame2018pontofrio} \cite{2017recfacial}.

Este trabalho explora um algoritmo de grande impacto na última década, o algoritmo Viola-Jones, e foi motivado pela implementação de um sistema de reconhecimento facial em uma rede de lojas de varejo na cidade do Rio de Janeiro.

\section{Objetivos}

O objetivo deste trabalho é fornecer uma base teórica para uma futura implantação de um sistema de reconhecimento facial, através do estudo do algoritmo de Viola-Jones.