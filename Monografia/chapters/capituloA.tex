\chapter{Detecção facial}

O papel de um detector de faces é, dada uma imagem arbitrária, determinar se ela contém faces humanas ou não e, caso positivo, retornar a localização e dimensões de cada face \cite{censtudy}.

A detecção de faces é utilizada em câmeras fotográficas para ajuste automático de foco, em softwares de imagens e redes sociais para marcação de pessoas e é uma etapa importante para o processo de reconhecimento facial. Antes de executar um algoritmo de reconhecimento facial, é de praxe realizar uma detecção facial a fim de concentrar os esforços do reconhecedor facial apenas nas áreas relevantes.

É preciso minimizar tanto a quantidade de faces não identificadas (falso-negativos) quanto objetos reconhecidos erroneamente como faces (falso-positivos) para obter uma performance aceitável. Para tanto, algoritmos de aprendizado de máquina podem ser muito úteis.

Diversas dificuldades influenciam na eficiência dos algoritmos, como ruídos, variação de iluminação, expressões faciais, imagem de fundo, orientação da cabeça, obstrução da face ou sobreposição de faces \cite{de2015processo} e, nota-se que, para análise de streamings de vídeo, é fundamental que a detecção facial seja realizada em tempo real.

Segundo, \citet{de2015processo}, "as técnicas mais citadas para realizar a detecção de faces são: casamento de padrões que consiste na detecção por meio de comparações com formas ométricas, modelos estatísticos, modelos baseado em redes neurais, modelos baseados em tons de pele e o Viola; Jones". Dentre os trabalhos anteriores a Viola-Jones, destacam-se \citet{rowley1998neural} e \citet{schneiderman2000statistical}.