% ----------------------------------------------------------
% RESUMOS
% ----------------------------------------------------------

% resumo em português
\setlength{\absparsep}{18pt} % ajusta o espaçamento dos parágrafos do resumo
\begin{resumo}
As inúmeras aplicações das técnicas para detecção e reconhecimento facial têm chamado muita atenção de empresas e governos. Esse crescente interesse pelo assunto atraiu investimentos em pesquisas e resultou em progressos significantes no desenvolvimento de novos métodos, bibliotecas, produtos e serviços.

Apesar de muitas dessas ferramentas serem descritas como simples e utilizáveis sem a necessidade de conhecimentos prévios em visão computacional, um embasamento teórico permite escolher as tecnologias apropriadas e usá-las de forma otimizada, considerando suas capacidades e limitações.

Este trabalho introduz as áreas de detecção e reconhecimento facial através de uma extensa revisão bibliográfica, que fornece uma visão geral sobre inúmeros métodos encontrados na literatura e apresenta uma coletânea de recursos úteis ao treinamento de classificadores e validação dos algoritmos.

Também é feito um estudo mais aprofundado sobre Viola-Jones e Eigenfaces ao apresentar o projeto e a implementação de um sistema capaz de detectar e reconhecer faces construído através da combinação desses métodos. É mostrado que a taxa de detecção do Eigenfaces não é suficiente para o uso desejado e são propostas alternativas.

O primeiro módulo desse sistema é executado em um Raspberry Pi e é um exemplo de como aliar conhecimento teórico, bibliotecas open source, ferramentas comerciais e hardware para a criação de um produto lucrativo.

 \textbf{Palavras-chaves}: Detecção facial. Viola-Jones. Reconhecimento facial. Eigenfaces. Raspberry Pi.
\end{resumo}

% resumo em inglês
\begin{resumo}[Abstract]
 \begin{otherlanguage*}{english}

The numerous applications of facial detection and recognition techniques have attracted much attention from companies and governments. This growing interest in the subject attracted investment in research and resulted in significant advances in the development of new methods, libraries, products and services.

Although many of these tools are described as simple and usable without the need for prior knowledge in computer vision, a theoretical basis allows one to choose the appropriate technologies and use them optimally, considering their capabilities and limitations.

This work introduces the areas of facial detection and recognition through an extensive bibliographic review, which provides an overview of the numerous methods found in the literature and presents a collection of useful resources for the training of classifiers and validation of algorithms.

Further study is also made on Viola-Jones and Eigenfaces while presenting the design and implementation of a system capable of detecting and recognizing faces constructed by combining these methods. It is shown that the detection rate of the Eigenfaces is not sufficient for the desired use and alternative solutions are proposed.

The first module of this system runs on a Raspberry Pi and is an example of how to combine theoretical knowledge, open source libraries, comercial tools and hardware in the creation of a profitable product.
   \vspace{\onelineskip}
 
   \noindent 
   \textbf{Key-words}: Facial detection. Viola-Jones. Facial recognition. Eigenfaces. Raspberry Pi.
 \end{otherlanguage*}
\end{resumo}

% resumo em francês 
%\begin{resumo}[Résumé]
% \begin{otherlanguage*}{french}
%    Il s'agit d'un résumé en français.
 
%   \textbf{Mots-clés}: latex. abntex. publication de textes.
% \end{otherlanguage*}
%\end{resumo}

% resumo em espanhol
%\begin{resumo}[Resumen]
% \begin{otherlanguage*}{spanish}
%   Este es el resumen en español.
  
%   \textbf{Palabras clave}: latex. abntex. publicación de textos.
% \end{otherlanguage*}
%\end{resumo}

