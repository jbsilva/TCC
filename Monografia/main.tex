%A classe é a dcc-nce, e o parâmetro a ser informado é diss (para dissertação de mestrado)
\documentclass[diss]{dcc-nce}
\usepackage[T1]{fontenc}
\usepackage{color,graphicx}
\usepackage{graphics}
\usepackage{url}
\usepackage{adjustbox}
\usepackage{amsmath,amssymb}
\usepackage{siunitx}
\usepackage{subcaption}
\usepackage{tikz}
\usetikzlibrary{arrows,automata,calc,fit,matrix,positioning,shadows,shapes}

%%%%% Tabelas com Tikz %%%%%%%%%%%%%%%%%%%%%%%%%%%%%%%
\tikzset{square matrix/.style={
    matrix of nodes,
    column sep=-\pgflinewidth, row sep=-\pgflinewidth,
    nodes={draw,
      minimum height=#1,
      anchor=center,
      text width=#1,
      align=center,
      inner sep=0pt
    },
  },
  square matrix/.default=2em
}
%%%%%%%%%%%%%%%%%%%%%%%%%%%%%%%%%%%%%%%%%%%%%%%%%%%%%%

%Para citar textualmente ao invés de com números, basta
% escrever \usepackage{natbib} ao invés da linha abaixo.
\usepackage[numbers]{natbib}

\usepackage{dsfont} %Usado para conjuntos N, Z, Q, R, C
\usepackage[portuguese,algoruled,longend]{algorithm2e}
\usepackage{algorithmic}
\usepackage[utf8]{inputenc}
\usepackage{listings}%Para inserir codigos fontes de programas no apendice.
\usepackage{xcolor}
% Definindo novas cores
\definecolor{verde}{rgb}{0,0.5,0}
% Configurando layout para mostrar codigos C++
\usepackage{listings}
\lstset{
  language=C++,
  basicstyle=\ttfamily\small,
  keywordstyle=\color{blue},
  stringstyle=\color{verde},
  commentstyle=\color{red},
  extendedchars=true,
  showspaces=false,
  showstringspaces=false,
  numbers=left,
  numberstyle=\tiny,
  breaklines=true,
  backgroundcolor=\color{green!10},
  breakautoindent=true,
  captionpos=b,
  xleftmargin=0pt,
}

% Para contagem do numero total de folhas:
\usepackage{everyshi}
\makeatletter
\let\totalpages\relax
\newcounter{mypage}
\EveryShipout{\stepcounter{mypage}}
\AtEndDocument{\clearpage
   \immediate\write\@auxout{%
    \string\gdef\string\totalpages{\themypage}}}
\makeatother


\topmargin=0in
\textheight=20.5cm


\begin{document}

%Este tem que vir primeiro neste arquivo, caso contrario nao aparecerao
%as palavras-chave na ficha catalografica:
\keyword{Detecção facial}
\keyword{Reconhecimento facial}
\keyword{Viola-Jones}
    % Editar o arquivo palavrasChavePortugues

%O restante vem depois:
% ----------------------------------------------------------
% CAPA E FOLHA DE ROSTO
% ----------------------------------------------------------
\titulo{Explorando o algoritmo de Viola-Jones na detecção e reconhecimento facial}
\autor{Julio Batista Silva}
\local{São Carlos, Brasil}
\data{2018}
\orientador{Prof.~Dr.~ Alexandre Luis Magalhães Levada}
\coorientador{}
\instituicao{%
  Universidade Federal de São Carlos -- UFSCar
  \par
  Departamento de Computação
  \par
  Engenharia de Computação}
\tipotrabalho{Trabalho de Conclusão de Curso}
% O preambulo deve conter o tipo do trabalho, o objetivo, 
% o nome da instituição e a área de concentração 
\preambulo{Trabalho de conclusão de curso.}



\pagenumbering{roman} %numeração de páginas em romano começa a partir do primeiro capítulo

\begin{dedicatoria}
    Dedico este trabalho aos meus pais, Cesar de Souza e Silva e Fátima Aparecida Batista Silva, por todo o amor, incentivo aos estudos e apoio, que foram fundamentais para a realização deste trabalho.
\end{dedicatoria}

\begin{agradecimentos}
    Agradeço aos meus pais, Cesar e Fátima, que sempre me apoiaram e me ajudaram durante a graduação.
    
    À minha companheira, Louise Lobão, pelo amor e carinho, que me deram força para superar as dificuldades da vida.
    
    Ao meu orientador, Alexandre Levada, pelo apoio, atenção, paciência e conselhos.
    
    À Universidade Federal de São Carlos por oferecer recursos e acesso a grandes professores, que foram fundamentais para a minha formação.
    
    Aos meus veteranos, calouros e colegas de curso que, de alguma forma, contribuíram para a minha graduação.
    
    Aos amigos que fiz durante o intercâmbio na Alemanha.
    
    À CAPES pela bolsa que me sustentou durante o intercâmbio.
    
    Ao meu gestor, Felipe Silva, por ter sugerido o tema deste trabalho e me ajudado a equilibrar meu tempo entre trabalho e estudo.
    
    A todos da Visagio.
    
    Ao pessoal do Dragões de Garagem.
    
    Aos meus gatos, Amélia e Joaquim, que me fizeram companhia durante a escrita desta monografia.
\end{agradecimentos}

\begin{abstract}
Este trabalho explora o algoritmo de Viola-Jones para detecção e reconhecimento facial através de uma revisão bibliográfica.
\end{abstract}
\begin{englishabstract}{}{Face detection, Face recognition, Viola-Jones}
This work...
\end{englishabstract}

\listoffigures{}

\listoftables{}

% Examples

\begin{listofabbrv}{XXXXXX}
  \item [LBP] Local Binary Patterns
\end{listofabbrv}
    % Editar o arquivo listaAbreviaturaSiglas.tex

\tableofcontents{}                  % Sumário

\parindent=1.25cm %start for each paragraph from the left margin
\parskip=20pt
\baselineskip=20pt

\pagenumbering{arabic} %numeração de páginas em arábico começa a partir do primeiro capítulo

\chapter{Introdução}

A capacidade de identificar faces e suas emoções é um importante mecanismo neurológico para interações sociais que, em certo grau, está presente até mesmo em recém nascidos \cite{morton1991conspec} \cite{fantz1961origin}.

Enquanto humanos possuem uma notável capacidade de reconhecer faces, o desenvolvimento de sistemas computacionais com capacidade similar é uma área de pesquisa em andamento há mais de cinco décadas \cite{bledsoe1964facial} \cite{chan1965man} \cite{bledsoe1966man} \cite{bledsoe1966model} \cite{boyer1991biographical} \cite{kelly1970visual} \cite{kanade1973picture}.

O desenvolvimento de modelos computacionais para reconhecimento facial são de interesse para diversas aplicações práticas como identificação criminal, sistemas de segurança, biometria, processamento de imagens e interação humano-computador.

Infelizmente, o desenvolvimento de um sistema computacional para reconhecimento facial automatizado é bastante difícil por diversos motivos. Faces são complexas, multidimensionais e expressivas \cite{turk1991eigenfaces} e as imagens podem sofrer variações em escala, localização, ponto de visão, iluminação e obstrução \cite{censtudy}.

\section{Motivação}

O desenvolvimento de novos algoritmos de processamento de imagens acompanhado do maior acesso a câmeras digitais, hardwares com alto poder de processamento, bibliotecas para visão computacional, como a OpenCV, e APIs de análise de imagem, como o Google Vision, o Microsoft Cognitive Services, o Amazon Rekognition e o IBM Watson Visual Recognition, tornou viável a implantação de sistemas com detecção e reconhecimento facial em diversas empresas e produtos.

Nos últimos anos, lojas de varejo têm usado com sucesso as tecnologias de detecção e reconhecimento facial para segurança, personalização de serviço, marketing e análise de sentimento \cite{fortune2015walmart} \cite{exame2018pontofrio} \cite{2017recfacial}.

Este trabalho explora um algoritmo de grande impacto na última década, o algoritmo Viola-Jones, e foi motivado pela implementação de um sistema de reconhecimento facial em uma rede de lojas de varejo na cidade do Rio de Janeiro.

\section{Objetivos}

O objetivo deste trabalho é fornecer uma base teórica para uma futura implantação de um sistema de reconhecimento facial, através do estudo do algoritmo de Viola-Jones.%
\chapter{Detecção facial}

O papel de um detector de faces é, dada uma imagem arbitrária, determinar se ela contém faces humanas ou não e, caso positivo, retornar a localização e dimensões de cada face \cite{censtudy}.

A detecção de faces é utilizada em câmeras fotográficas para ajuste automático de foco, em softwares de imagens e redes sociais para marcação de pessoas e é uma etapa importante para o processo de reconhecimento facial. Antes de executar um algoritmo de reconhecimento facial, é de praxe realizar uma detecção facial a fim de concentrar os esforços do reconhecedor facial apenas nas áreas relevantes.

É preciso minimizar tanto a quantidade de faces não identificadas (falso-negativos) quanto objetos reconhecidos erroneamente como faces (falso-positivos) para obter uma performance aceitável. Para tanto, algoritmos de aprendizado de máquina podem ser muito úteis.

Diversas dificuldades influenciam na eficiência dos algoritmos, como ruídos, variação de iluminação, expressões faciais, imagem de fundo, orientação da cabeça, obstrução da face ou sobreposição de faces \cite{de2015processo} e, nota-se que, para análise de streamings de vídeo, é fundamental que a detecção facial seja realizada em tempo real.

Segundo, \citet{de2015processo}, "as técnicas mais citadas para realizar a detecção de faces são: casamento de padrões que consiste na detecção por meio de comparações com formas ométricas, modelos estatísticos, modelos baseado em redes neurais, modelos baseados em tons de pele e o Viola; Jones". Dentre os trabalhos anteriores a Viola-Jones, destacam-se \citet{rowley1998neural} e \citet{schneiderman2000statistical}.%
\input{chapters/capituloB}%
% ----------------------------------------------------------
% Conclusão (outro exemplo de capítulo sem numeração e presente no sumário)
% ----------------------------------------------------------
% \chapter*[Conclusão]{Conclusão}
% \addcontentsline{toc}{chapter}{Conclusão}
% ----------------------------------------------------------

O presente trabalho apresentou uma visão ampla dos conceitos e métodos de detecção e reconhecimento facial e explicações detalhadas sobre dois algoritmos, incluindo a implementação de sistemas computacionais capazes de detectar e reconhecer faces utilizando esses algoritmos.

Para alcançar o objetivo de fornecer uma base teórica para a implantação de um sistema de detecção e reconhecimento facial, foi necessária uma extensa pesquisa bibliográfica e a conclusão de cursos sobre visão computacional.

Ficou claro que reconhecimento facial irrestrito ainda é um desafio na área de processamento de imagens e precisa superar diversas dificuldades geradas por fatores como iluminação, baixa resolução, variação de pose, oclusão e expressão facial.

Foi possível mostrar a validade do algoritmo de Viola e Jones utilizando a biblioteca OpenCV.

Dificuldades do treinamento: determinar taxas de detecção e falso-positivo, determinar número de estágios, tempo de treinamento x performance do detector.

\section{Sugestão de melhorias e trabalhos futuros}

Treinar um novo detector utilizando mais imagens positivas e negativas melhor selecionadas.

Usar valor mais baixo para taxa máxima de falso-positivo (maxFalseAlarmRate)

Variar taxas de detecção e falso-positivo por estágio da cascata.

Usar mais estágios

Paralelização do processo

Uso de arquitetura na nuvem como Amazon EC2

Imprimir case em 3D para Raspberry Pi

Utilizar algoritmos baseados em redes neurais ou soluções comerciais%

\bibliography{post-text/referencias}       % Editar o arquivo "referencias.bib"
\bibliographystyle{latex-stuff/abnt-ufrgs} % Procura pelo arquivo "abnt-ufrgs" - normas ABNT.

\clearpage

\appendix
\input{post-text/apendice}          % Editar o arquivo apendice.tex

\end{document}