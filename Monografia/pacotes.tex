% ----------------------------------------------------------
% PACOTES BÁSICOS
% ----------------------------------------------------------
\usepackage{lmodern}            % Usa a fonte Latin Modern          
\usepackage[T1]{fontenc}        % Selecao de codigos de fonte.
\usepackage[utf8]{inputenc}     % Codificacao do documento (conversão automática dos acentos)
\usepackage{lastpage}           % Usado pela Ficha catalográfica
\usepackage{indentfirst}        % Indenta o primeiro parágrafo de cada seção.
\usepackage{color}              % Controle das cores
\usepackage{graphicx}           % Inclusão de gráficos
\usepackage{microtype}          % para melhorias de justificação
\usepackage[newfloat]{minted}   % Pygments
\usepackage{nomencl}            % Necessário para o commando makeindex
\usepackage[brazilian,hyperpageref]{backref}    % Paginas com as citações na bibl
\usepackage[alf]{abntex2cite}   % Citações padrão ABNT
\usepackage{lipsum}             
\usepackage{adjustbox}
\usepackage{amsmath,amssymb}
\usepackage{siunitx}
\usepackage{booktabs}
\usepackage{longtable}
\usepackage{tabu}
\usepackage{multirow}
\usepackage{subcaption}
%\usepackage{epigraph}
\usepackage{lscape}
\usepackage{xurl}               % Quebra links longos direito
\usepackage[ruled,vlined,portuguese,onelanguage]{algorithm2e} %for psuedo code
\usepackage{tikz}
\usetikzlibrary{arrows,automata,calc,chains,fit,matrix,positioning,quotes,shadows,shapes}
\usepackage{pgfplots}
\pgfplotsset{compat=newest,compat/show suggested version=false}

% CONFIGURAÇÕES DE PACOTES
% Configurações do pacote backref
% Usado sem a opção hyperpageref de backref
\renewcommand{\backrefpagesname}{Citado na(s) página(s):~}
% Texto padrão antes do número das páginas
\renewcommand{\backref}{}
% Define os textos da citação
\renewcommand*{\backrefalt}[4]{
    \ifcase #1 %
        Nenhuma citação no texto.%
    \or
        Citado na página #2.%
    \else
        Citado #1 vezes nas páginas #2.%
    \fi}%

% Corrige bug do anexo (https://github.com/abntex/abntex2/issues/76)
\newcommand{\refanexo}[1]{\hyperref[#1]{Anexo~\ref{#1}}}

%%%%%%%%%%%%%%%%%%%%%%%%%%%%%%%%%%%%%%%%%%%%%%%%%%%%%%%%%%%%%%%%%%%%%%%%
%% TIKZ
%%%%%%%%%%%%%%%%%%%%%%%%%%%%%%%%%%%%%%%%%%%%%%%%%%%%%%%%%%%%%%%%%%%%%%%%
% Tabelas
\tikzset{square matrix/.style={
    matrix of nodes,
    column sep=-\pgflinewidth, row sep=-\pgflinewidth,
    nodes={draw,
      minimum height=#1,
      anchor=center,
      text width=#1,
      align=center,
      inner sep=0pt
    },
  },
  square matrix/.default=2em
}

% Distância entre extremidades de dois nós
\makeatletter
\def\DistanciaExtremidades(#1,#2)#3{%
\pgfpointdiff{\pgfpointanchor{#1}{west}}{\pgfpointanchor{#2}{east}}
\pgfmathsetmacro{\myheight}{veclen(\pgf@x,\pgf@y)}
\global\expandafter\edef\csname #3\endcsname{\myheight}
}
\makeatother

% Distância entre centros de dois nós
\makeatletter
\def\DistanciaCentros(#1,#2)#3{%
\pgfpointdiff{\pgfpointanchor{#1}{center}}{\pgfpointanchor{#2}{center}}
\pgfmathsetmacro{\myheight}{veclen(\pgf@x,\pgf@y)}
\global\expandafter\edef\csname #3\endcsname{\myheight}
}
\makeatother

%%%%%%%%%%%%%%%%%%%%%%%%%%%%%%%%%%%%%%%%%%%%%%%%%%%%%%%%%%%%%%%%%%%%%%%%
%% LISTA DE CÓDIGOS
%%%%%%%%%%%%%%%%%%%%%%%%%%%%%%%%%%%%%%%%%%%%%%%%%%%%%%%%%%%%%%%%%%%%%%%%
\newenvironment{code}{\captionsetup{type=listing}}{}

\makeatletter
\let\l@listing\l@figure
\def\newfloat@listoflisting@hook{\let\figurename\listingname}
\makeatother

\SetupFloatingEnvironment{listing}{%
  fileext=lol,
  listname={Lista de códigos},
  name=Código,
  placement=p,
  within=none,
  chapterlistsgaps=on}

\setminted{% bgcolor = gray!15, 
            frame = lines,
            mathescape,
            autogobble,
            %breakanywhere,
            breaklines,
            framesep = 2mm,
            baselinestretch = 1.2,
            fontsize = \footnotesize,
            linenos
}

%%%%%%%%%%%%%%%%%%%%%%%%%%%%%%%%%%%%%%%%%%%%%%%%%%%%%%%%%%%%%%%%%%%%%%%%
%% LISTA DE ALGORITMOS
%%%%%%%%%%%%%%%%%%%%%%%%%%%%%%%%%%%%%%%%%%%%%%%%%%%%%%%%%%%%%%%%%%%%%%%%
%\let\oldlistofalgorithms\listofalgorithms
%\let\oldnumberline\numberline%
%\newcommand{\algnumberline}[1]{Algoritmo~#1 -- }
%\renewcommand{\listofalgorithms}{%
%  \let\numberline\algnumberline%
%  \oldlistofalgorithms
%  \let\numberline\oldnumberline%
%}
